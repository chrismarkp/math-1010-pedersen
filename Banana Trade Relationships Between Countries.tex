\documentclass[11pt]{article}

    \usepackage[breakable]{tcolorbox}
    \usepackage{parskip} % Stop auto-indenting (to mimic markdown behaviour)
    

    % Basic figure setup, for now with no caption control since it's done
    % automatically by Pandoc (which extracts ![](path) syntax from Markdown).
    \usepackage{graphicx}
    % Maintain compatibility with old templates. Remove in nbconvert 6.0
    \let\Oldincludegraphics\includegraphics
    % Ensure that by default, figures have no caption (until we provide a
    % proper Figure object with a Caption API and a way to capture that
    % in the conversion process - todo).
    \usepackage{caption}
    \DeclareCaptionFormat{nocaption}{}
    \captionsetup{format=nocaption,aboveskip=0pt,belowskip=0pt}

    \usepackage{float}
    \floatplacement{figure}{H} % forces figures to be placed at the correct location
    \usepackage{xcolor} % Allow colors to be defined
    \usepackage{enumerate} % Needed for markdown enumerations to work
    \usepackage{geometry} % Used to adjust the document margins
    \usepackage{amsmath} % Equations
    \usepackage{amssymb} % Equations
    \usepackage{textcomp} % defines textquotesingle
    % Hack from http://tex.stackexchange.com/a/47451/13684:
    \AtBeginDocument{%
        \def\PYZsq{\textquotesingle}% Upright quotes in Pygmentized code
    }
    \usepackage{upquote} % Upright quotes for verbatim code
    \usepackage{eurosym} % defines \euro

    \usepackage{iftex}
    \ifPDFTeX
        \usepackage[T1]{fontenc}
        \IfFileExists{alphabeta.sty}{
              \usepackage{alphabeta}
          }{
              \usepackage[mathletters]{ucs}
              \usepackage[utf8x]{inputenc}
          }
    \else
        \usepackage{fontspec}
        \usepackage{unicode-math}
    \fi

    \usepackage{fancyvrb} % verbatim replacement that allows latex
    \usepackage{grffile} % extends the file name processing of package graphics
                         % to support a larger range
    \makeatletter % fix for old versions of grffile with XeLaTeX
    \@ifpackagelater{grffile}{2019/11/01}
    {
      % Do nothing on new versions
    }
    {
      \def\Gread@@xetex#1{%
        \IfFileExists{"\Gin@base".bb}%
        {\Gread@eps{\Gin@base.bb}}%
        {\Gread@@xetex@aux#1}%
      }
    }
    \makeatother
    \usepackage[Export]{adjustbox} % Used to constrain images to a maximum size
    \adjustboxset{max size={0.9\linewidth}{0.9\paperheight}}

    % The hyperref package gives us a pdf with properly built
    % internal navigation ('pdf bookmarks' for the table of contents,
    % internal cross-reference links, web links for URLs, etc.)
    \usepackage{hyperref}
    % The default LaTeX title has an obnoxious amount of whitespace. By default,
    % titling removes some of it. It also provides customization options.
    \usepackage{titling}
    \usepackage{longtable} % longtable support required by pandoc >1.10
    \usepackage{booktabs}  % table support for pandoc > 1.12.2
    \usepackage{array}     % table support for pandoc >= 2.11.3
    \usepackage{calc}      % table minipage width calculation for pandoc >= 2.11.1
    \usepackage[inline]{enumitem} % IRkernel/repr support (it uses the enumerate* environment)
    \usepackage[normalem]{ulem} % ulem is needed to support strikethroughs (\sout)
                                % normalem makes italics be italics, not underlines
    \usepackage{mathrsfs}
    

    
    % Colors for the hyperref package
    \definecolor{urlcolor}{rgb}{0,.145,.698}
    \definecolor{linkcolor}{rgb}{.71,0.21,0.01}
    \definecolor{citecolor}{rgb}{.12,.54,.11}

    % ANSI colors
    \definecolor{ansi-black}{HTML}{3E424D}
    \definecolor{ansi-black-intense}{HTML}{282C36}
    \definecolor{ansi-red}{HTML}{E75C58}
    \definecolor{ansi-red-intense}{HTML}{B22B31}
    \definecolor{ansi-green}{HTML}{00A250}
    \definecolor{ansi-green-intense}{HTML}{007427}
    \definecolor{ansi-yellow}{HTML}{DDB62B}
    \definecolor{ansi-yellow-intense}{HTML}{B27D12}
    \definecolor{ansi-blue}{HTML}{208FFB}
    \definecolor{ansi-blue-intense}{HTML}{0065CA}
    \definecolor{ansi-magenta}{HTML}{D160C4}
    \definecolor{ansi-magenta-intense}{HTML}{A03196}
    \definecolor{ansi-cyan}{HTML}{60C6C8}
    \definecolor{ansi-cyan-intense}{HTML}{258F8F}
    \definecolor{ansi-white}{HTML}{C5C1B4}
    \definecolor{ansi-white-intense}{HTML}{A1A6B2}
    \definecolor{ansi-default-inverse-fg}{HTML}{FFFFFF}
    \definecolor{ansi-default-inverse-bg}{HTML}{000000}

    % common color for the border for error outputs.
    \definecolor{outerrorbackground}{HTML}{FFDFDF}

    % commands and environments needed by pandoc snippets
    % extracted from the output of `pandoc -s`
    \providecommand{\tightlist}{%
      \setlength{\itemsep}{0pt}\setlength{\parskip}{0pt}}
    \DefineVerbatimEnvironment{Highlighting}{Verbatim}{commandchars=\\\{\}}
    % Add ',fontsize=\small' for more characters per line
    \newenvironment{Shaded}{}{}
    \newcommand{\KeywordTok}[1]{\textcolor[rgb]{0.00,0.44,0.13}{\textbf{{#1}}}}
    \newcommand{\DataTypeTok}[1]{\textcolor[rgb]{0.56,0.13,0.00}{{#1}}}
    \newcommand{\DecValTok}[1]{\textcolor[rgb]{0.25,0.63,0.44}{{#1}}}
    \newcommand{\BaseNTok}[1]{\textcolor[rgb]{0.25,0.63,0.44}{{#1}}}
    \newcommand{\FloatTok}[1]{\textcolor[rgb]{0.25,0.63,0.44}{{#1}}}
    \newcommand{\CharTok}[1]{\textcolor[rgb]{0.25,0.44,0.63}{{#1}}}
    \newcommand{\StringTok}[1]{\textcolor[rgb]{0.25,0.44,0.63}{{#1}}}
    \newcommand{\CommentTok}[1]{\textcolor[rgb]{0.38,0.63,0.69}{\textit{{#1}}}}
    \newcommand{\OtherTok}[1]{\textcolor[rgb]{0.00,0.44,0.13}{{#1}}}
    \newcommand{\AlertTok}[1]{\textcolor[rgb]{1.00,0.00,0.00}{\textbf{{#1}}}}
    \newcommand{\FunctionTok}[1]{\textcolor[rgb]{0.02,0.16,0.49}{{#1}}}
    \newcommand{\RegionMarkerTok}[1]{{#1}}
    \newcommand{\ErrorTok}[1]{\textcolor[rgb]{1.00,0.00,0.00}{\textbf{{#1}}}}
    \newcommand{\NormalTok}[1]{{#1}}

    % Additional commands for more recent versions of Pandoc
    \newcommand{\ConstantTok}[1]{\textcolor[rgb]{0.53,0.00,0.00}{{#1}}}
    \newcommand{\SpecialCharTok}[1]{\textcolor[rgb]{0.25,0.44,0.63}{{#1}}}
    \newcommand{\VerbatimStringTok}[1]{\textcolor[rgb]{0.25,0.44,0.63}{{#1}}}
    \newcommand{\SpecialStringTok}[1]{\textcolor[rgb]{0.73,0.40,0.53}{{#1}}}
    \newcommand{\ImportTok}[1]{{#1}}
    \newcommand{\DocumentationTok}[1]{\textcolor[rgb]{0.73,0.13,0.13}{\textit{{#1}}}}
    \newcommand{\AnnotationTok}[1]{\textcolor[rgb]{0.38,0.63,0.69}{\textbf{\textit{{#1}}}}}
    \newcommand{\CommentVarTok}[1]{\textcolor[rgb]{0.38,0.63,0.69}{\textbf{\textit{{#1}}}}}
    \newcommand{\VariableTok}[1]{\textcolor[rgb]{0.10,0.09,0.49}{{#1}}}
    \newcommand{\ControlFlowTok}[1]{\textcolor[rgb]{0.00,0.44,0.13}{\textbf{{#1}}}}
    \newcommand{\OperatorTok}[1]{\textcolor[rgb]{0.40,0.40,0.40}{{#1}}}
    \newcommand{\BuiltInTok}[1]{{#1}}
    \newcommand{\ExtensionTok}[1]{{#1}}
    \newcommand{\PreprocessorTok}[1]{\textcolor[rgb]{0.74,0.48,0.00}{{#1}}}
    \newcommand{\AttributeTok}[1]{\textcolor[rgb]{0.49,0.56,0.16}{{#1}}}
    \newcommand{\InformationTok}[1]{\textcolor[rgb]{0.38,0.63,0.69}{\textbf{\textit{{#1}}}}}
    \newcommand{\WarningTok}[1]{\textcolor[rgb]{0.38,0.63,0.69}{\textbf{\textit{{#1}}}}}


    % Define a nice break command that doesn't care if a line doesn't already
    % exist.
    \def\br{\hspace*{\fill} \\* }
    % Math Jax compatibility definitions
    \def\gt{>}
    \def\lt{<}
    \let\Oldtex\TeX
    \let\Oldlatex\LaTeX
    \renewcommand{\TeX}{\textrm{\Oldtex}}
    \renewcommand{\LaTeX}{\textrm{\Oldlatex}}
    % Document parameters
    % Document title
    \title{Banana Trade Relationships Between Countries}
    
    
    
    
    
% Pygments definitions
\makeatletter
\def\PY@reset{\let\PY@it=\relax \let\PY@bf=\relax%
    \let\PY@ul=\relax \let\PY@tc=\relax%
    \let\PY@bc=\relax \let\PY@ff=\relax}
\def\PY@tok#1{\csname PY@tok@#1\endcsname}
\def\PY@toks#1+{\ifx\relax#1\empty\else%
    \PY@tok{#1}\expandafter\PY@toks\fi}
\def\PY@do#1{\PY@bc{\PY@tc{\PY@ul{%
    \PY@it{\PY@bf{\PY@ff{#1}}}}}}}
\def\PY#1#2{\PY@reset\PY@toks#1+\relax+\PY@do{#2}}

\@namedef{PY@tok@w}{\def\PY@tc##1{\textcolor[rgb]{0.73,0.73,0.73}{##1}}}
\@namedef{PY@tok@c}{\let\PY@it=\textit\def\PY@tc##1{\textcolor[rgb]{0.24,0.48,0.48}{##1}}}
\@namedef{PY@tok@cp}{\def\PY@tc##1{\textcolor[rgb]{0.61,0.40,0.00}{##1}}}
\@namedef{PY@tok@k}{\let\PY@bf=\textbf\def\PY@tc##1{\textcolor[rgb]{0.00,0.50,0.00}{##1}}}
\@namedef{PY@tok@kp}{\def\PY@tc##1{\textcolor[rgb]{0.00,0.50,0.00}{##1}}}
\@namedef{PY@tok@kt}{\def\PY@tc##1{\textcolor[rgb]{0.69,0.00,0.25}{##1}}}
\@namedef{PY@tok@o}{\def\PY@tc##1{\textcolor[rgb]{0.40,0.40,0.40}{##1}}}
\@namedef{PY@tok@ow}{\let\PY@bf=\textbf\def\PY@tc##1{\textcolor[rgb]{0.67,0.13,1.00}{##1}}}
\@namedef{PY@tok@nb}{\def\PY@tc##1{\textcolor[rgb]{0.00,0.50,0.00}{##1}}}
\@namedef{PY@tok@nf}{\def\PY@tc##1{\textcolor[rgb]{0.00,0.00,1.00}{##1}}}
\@namedef{PY@tok@nc}{\let\PY@bf=\textbf\def\PY@tc##1{\textcolor[rgb]{0.00,0.00,1.00}{##1}}}
\@namedef{PY@tok@nn}{\let\PY@bf=\textbf\def\PY@tc##1{\textcolor[rgb]{0.00,0.00,1.00}{##1}}}
\@namedef{PY@tok@ne}{\let\PY@bf=\textbf\def\PY@tc##1{\textcolor[rgb]{0.80,0.25,0.22}{##1}}}
\@namedef{PY@tok@nv}{\def\PY@tc##1{\textcolor[rgb]{0.10,0.09,0.49}{##1}}}
\@namedef{PY@tok@no}{\def\PY@tc##1{\textcolor[rgb]{0.53,0.00,0.00}{##1}}}
\@namedef{PY@tok@nl}{\def\PY@tc##1{\textcolor[rgb]{0.46,0.46,0.00}{##1}}}
\@namedef{PY@tok@ni}{\let\PY@bf=\textbf\def\PY@tc##1{\textcolor[rgb]{0.44,0.44,0.44}{##1}}}
\@namedef{PY@tok@na}{\def\PY@tc##1{\textcolor[rgb]{0.41,0.47,0.13}{##1}}}
\@namedef{PY@tok@nt}{\let\PY@bf=\textbf\def\PY@tc##1{\textcolor[rgb]{0.00,0.50,0.00}{##1}}}
\@namedef{PY@tok@nd}{\def\PY@tc##1{\textcolor[rgb]{0.67,0.13,1.00}{##1}}}
\@namedef{PY@tok@s}{\def\PY@tc##1{\textcolor[rgb]{0.73,0.13,0.13}{##1}}}
\@namedef{PY@tok@sd}{\let\PY@it=\textit\def\PY@tc##1{\textcolor[rgb]{0.73,0.13,0.13}{##1}}}
\@namedef{PY@tok@si}{\let\PY@bf=\textbf\def\PY@tc##1{\textcolor[rgb]{0.64,0.35,0.47}{##1}}}
\@namedef{PY@tok@se}{\let\PY@bf=\textbf\def\PY@tc##1{\textcolor[rgb]{0.67,0.36,0.12}{##1}}}
\@namedef{PY@tok@sr}{\def\PY@tc##1{\textcolor[rgb]{0.64,0.35,0.47}{##1}}}
\@namedef{PY@tok@ss}{\def\PY@tc##1{\textcolor[rgb]{0.10,0.09,0.49}{##1}}}
\@namedef{PY@tok@sx}{\def\PY@tc##1{\textcolor[rgb]{0.00,0.50,0.00}{##1}}}
\@namedef{PY@tok@m}{\def\PY@tc##1{\textcolor[rgb]{0.40,0.40,0.40}{##1}}}
\@namedef{PY@tok@gh}{\let\PY@bf=\textbf\def\PY@tc##1{\textcolor[rgb]{0.00,0.00,0.50}{##1}}}
\@namedef{PY@tok@gu}{\let\PY@bf=\textbf\def\PY@tc##1{\textcolor[rgb]{0.50,0.00,0.50}{##1}}}
\@namedef{PY@tok@gd}{\def\PY@tc##1{\textcolor[rgb]{0.63,0.00,0.00}{##1}}}
\@namedef{PY@tok@gi}{\def\PY@tc##1{\textcolor[rgb]{0.00,0.52,0.00}{##1}}}
\@namedef{PY@tok@gr}{\def\PY@tc##1{\textcolor[rgb]{0.89,0.00,0.00}{##1}}}
\@namedef{PY@tok@ge}{\let\PY@it=\textit}
\@namedef{PY@tok@gs}{\let\PY@bf=\textbf}
\@namedef{PY@tok@gp}{\let\PY@bf=\textbf\def\PY@tc##1{\textcolor[rgb]{0.00,0.00,0.50}{##1}}}
\@namedef{PY@tok@go}{\def\PY@tc##1{\textcolor[rgb]{0.44,0.44,0.44}{##1}}}
\@namedef{PY@tok@gt}{\def\PY@tc##1{\textcolor[rgb]{0.00,0.27,0.87}{##1}}}
\@namedef{PY@tok@err}{\def\PY@bc##1{{\setlength{\fboxsep}{\string -\fboxrule}\fcolorbox[rgb]{1.00,0.00,0.00}{1,1,1}{\strut ##1}}}}
\@namedef{PY@tok@kc}{\let\PY@bf=\textbf\def\PY@tc##1{\textcolor[rgb]{0.00,0.50,0.00}{##1}}}
\@namedef{PY@tok@kd}{\let\PY@bf=\textbf\def\PY@tc##1{\textcolor[rgb]{0.00,0.50,0.00}{##1}}}
\@namedef{PY@tok@kn}{\let\PY@bf=\textbf\def\PY@tc##1{\textcolor[rgb]{0.00,0.50,0.00}{##1}}}
\@namedef{PY@tok@kr}{\let\PY@bf=\textbf\def\PY@tc##1{\textcolor[rgb]{0.00,0.50,0.00}{##1}}}
\@namedef{PY@tok@bp}{\def\PY@tc##1{\textcolor[rgb]{0.00,0.50,0.00}{##1}}}
\@namedef{PY@tok@fm}{\def\PY@tc##1{\textcolor[rgb]{0.00,0.00,1.00}{##1}}}
\@namedef{PY@tok@vc}{\def\PY@tc##1{\textcolor[rgb]{0.10,0.09,0.49}{##1}}}
\@namedef{PY@tok@vg}{\def\PY@tc##1{\textcolor[rgb]{0.10,0.09,0.49}{##1}}}
\@namedef{PY@tok@vi}{\def\PY@tc##1{\textcolor[rgb]{0.10,0.09,0.49}{##1}}}
\@namedef{PY@tok@vm}{\def\PY@tc##1{\textcolor[rgb]{0.10,0.09,0.49}{##1}}}
\@namedef{PY@tok@sa}{\def\PY@tc##1{\textcolor[rgb]{0.73,0.13,0.13}{##1}}}
\@namedef{PY@tok@sb}{\def\PY@tc##1{\textcolor[rgb]{0.73,0.13,0.13}{##1}}}
\@namedef{PY@tok@sc}{\def\PY@tc##1{\textcolor[rgb]{0.73,0.13,0.13}{##1}}}
\@namedef{PY@tok@dl}{\def\PY@tc##1{\textcolor[rgb]{0.73,0.13,0.13}{##1}}}
\@namedef{PY@tok@s2}{\def\PY@tc##1{\textcolor[rgb]{0.73,0.13,0.13}{##1}}}
\@namedef{PY@tok@sh}{\def\PY@tc##1{\textcolor[rgb]{0.73,0.13,0.13}{##1}}}
\@namedef{PY@tok@s1}{\def\PY@tc##1{\textcolor[rgb]{0.73,0.13,0.13}{##1}}}
\@namedef{PY@tok@mb}{\def\PY@tc##1{\textcolor[rgb]{0.40,0.40,0.40}{##1}}}
\@namedef{PY@tok@mf}{\def\PY@tc##1{\textcolor[rgb]{0.40,0.40,0.40}{##1}}}
\@namedef{PY@tok@mh}{\def\PY@tc##1{\textcolor[rgb]{0.40,0.40,0.40}{##1}}}
\@namedef{PY@tok@mi}{\def\PY@tc##1{\textcolor[rgb]{0.40,0.40,0.40}{##1}}}
\@namedef{PY@tok@il}{\def\PY@tc##1{\textcolor[rgb]{0.40,0.40,0.40}{##1}}}
\@namedef{PY@tok@mo}{\def\PY@tc##1{\textcolor[rgb]{0.40,0.40,0.40}{##1}}}
\@namedef{PY@tok@ch}{\let\PY@it=\textit\def\PY@tc##1{\textcolor[rgb]{0.24,0.48,0.48}{##1}}}
\@namedef{PY@tok@cm}{\let\PY@it=\textit\def\PY@tc##1{\textcolor[rgb]{0.24,0.48,0.48}{##1}}}
\@namedef{PY@tok@cpf}{\let\PY@it=\textit\def\PY@tc##1{\textcolor[rgb]{0.24,0.48,0.48}{##1}}}
\@namedef{PY@tok@c1}{\let\PY@it=\textit\def\PY@tc##1{\textcolor[rgb]{0.24,0.48,0.48}{##1}}}
\@namedef{PY@tok@cs}{\let\PY@it=\textit\def\PY@tc##1{\textcolor[rgb]{0.24,0.48,0.48}{##1}}}

\def\PYZbs{\char`\\}
\def\PYZus{\char`\_}
\def\PYZob{\char`\{}
\def\PYZcb{\char`\}}
\def\PYZca{\char`\^}
\def\PYZam{\char`\&}
\def\PYZlt{\char`\<}
\def\PYZgt{\char`\>}
\def\PYZsh{\char`\#}
\def\PYZpc{\char`\%}
\def\PYZdl{\char`\$}
\def\PYZhy{\char`\-}
\def\PYZsq{\char`\'}
\def\PYZdq{\char`\"}
\def\PYZti{\char`\~}
% for compatibility with earlier versions
\def\PYZat{@}
\def\PYZlb{[}
\def\PYZrb{]}
\makeatother


    % For linebreaks inside Verbatim environment from package fancyvrb.
    \makeatletter
        \newbox\Wrappedcontinuationbox
        \newbox\Wrappedvisiblespacebox
        \newcommand*\Wrappedvisiblespace {\textcolor{red}{\textvisiblespace}}
        \newcommand*\Wrappedcontinuationsymbol {\textcolor{red}{\llap{\tiny$\m@th\hookrightarrow$}}}
        \newcommand*\Wrappedcontinuationindent {3ex }
        \newcommand*\Wrappedafterbreak {\kern\Wrappedcontinuationindent\copy\Wrappedcontinuationbox}
        % Take advantage of the already applied Pygments mark-up to insert
        % potential linebreaks for TeX processing.
        %        {, <, #, %, $, ' and ": go to next line.
        %        _, }, ^, &, >, - and ~: stay at end of broken line.
        % Use of \textquotesingle for straight quote.
        \newcommand*\Wrappedbreaksatspecials {%
            \def\PYGZus{\discretionary{\char`\_}{\Wrappedafterbreak}{\char`\_}}%
            \def\PYGZob{\discretionary{}{\Wrappedafterbreak\char`\{}{\char`\{}}%
            \def\PYGZcb{\discretionary{\char`\}}{\Wrappedafterbreak}{\char`\}}}%
            \def\PYGZca{\discretionary{\char`\^}{\Wrappedafterbreak}{\char`\^}}%
            \def\PYGZam{\discretionary{\char`\&}{\Wrappedafterbreak}{\char`\&}}%
            \def\PYGZlt{\discretionary{}{\Wrappedafterbreak\char`\<}{\char`\<}}%
            \def\PYGZgt{\discretionary{\char`\>}{\Wrappedafterbreak}{\char`\>}}%
            \def\PYGZsh{\discretionary{}{\Wrappedafterbreak\char`\#}{\char`\#}}%
            \def\PYGZpc{\discretionary{}{\Wrappedafterbreak\char`\%}{\char`\%}}%
            \def\PYGZdl{\discretionary{}{\Wrappedafterbreak\char`\$}{\char`\$}}%
            \def\PYGZhy{\discretionary{\char`\-}{\Wrappedafterbreak}{\char`\-}}%
            \def\PYGZsq{\discretionary{}{\Wrappedafterbreak\textquotesingle}{\textquotesingle}}%
            \def\PYGZdq{\discretionary{}{\Wrappedafterbreak\char`\"}{\char`\"}}%
            \def\PYGZti{\discretionary{\char`\~}{\Wrappedafterbreak}{\char`\~}}%
        }
        % Some characters . , ; ? ! / are not pygmentized.
        % This macro makes them "active" and they will insert potential linebreaks
        \newcommand*\Wrappedbreaksatpunct {%
            \lccode`\~`\.\lowercase{\def~}{\discretionary{\hbox{\char`\.}}{\Wrappedafterbreak}{\hbox{\char`\.}}}%
            \lccode`\~`\,\lowercase{\def~}{\discretionary{\hbox{\char`\,}}{\Wrappedafterbreak}{\hbox{\char`\,}}}%
            \lccode`\~`\;\lowercase{\def~}{\discretionary{\hbox{\char`\;}}{\Wrappedafterbreak}{\hbox{\char`\;}}}%
            \lccode`\~`\:\lowercase{\def~}{\discretionary{\hbox{\char`\:}}{\Wrappedafterbreak}{\hbox{\char`\:}}}%
            \lccode`\~`\?\lowercase{\def~}{\discretionary{\hbox{\char`\?}}{\Wrappedafterbreak}{\hbox{\char`\?}}}%
            \lccode`\~`\!\lowercase{\def~}{\discretionary{\hbox{\char`\!}}{\Wrappedafterbreak}{\hbox{\char`\!}}}%
            \lccode`\~`\/\lowercase{\def~}{\discretionary{\hbox{\char`\/}}{\Wrappedafterbreak}{\hbox{\char`\/}}}%
            \catcode`\.\active
            \catcode`\,\active
            \catcode`\;\active
            \catcode`\:\active
            \catcode`\?\active
            \catcode`\!\active
            \catcode`\/\active
            \lccode`\~`\~
        }
    \makeatother

    \let\OriginalVerbatim=\Verbatim
    \makeatletter
    \renewcommand{\Verbatim}[1][1]{%
        %\parskip\z@skip
        \sbox\Wrappedcontinuationbox {\Wrappedcontinuationsymbol}%
        \sbox\Wrappedvisiblespacebox {\FV@SetupFont\Wrappedvisiblespace}%
        \def\FancyVerbFormatLine ##1{\hsize\linewidth
            \vtop{\raggedright\hyphenpenalty\z@\exhyphenpenalty\z@
                \doublehyphendemerits\z@\finalhyphendemerits\z@
                \strut ##1\strut}%
        }%
        % If the linebreak is at a space, the latter will be displayed as visible
        % space at end of first line, and a continuation symbol starts next line.
        % Stretch/shrink are however usually zero for typewriter font.
        \def\FV@Space {%
            \nobreak\hskip\z@ plus\fontdimen3\font minus\fontdimen4\font
            \discretionary{\copy\Wrappedvisiblespacebox}{\Wrappedafterbreak}
            {\kern\fontdimen2\font}%
        }%

        % Allow breaks at special characters using \PYG... macros.
        \Wrappedbreaksatspecials
        % Breaks at punctuation characters . , ; ? ! and / need catcode=\active
        \OriginalVerbatim[#1,codes*=\Wrappedbreaksatpunct]%
    }
    \makeatother

    % Exact colors from NB
    \definecolor{incolor}{HTML}{303F9F}
    \definecolor{outcolor}{HTML}{D84315}
    \definecolor{cellborder}{HTML}{CFCFCF}
    \definecolor{cellbackground}{HTML}{F7F7F7}

    % prompt
    \makeatletter
    \newcommand{\boxspacing}{\kern\kvtcb@left@rule\kern\kvtcb@boxsep}
    \makeatother
    \newcommand{\prompt}[4]{
        {\ttfamily\llap{{\color{#2}[#3]:\hspace{3pt}#4}}\vspace{-\baselineskip}}
    }
    

    
    % Prevent overflowing lines due to hard-to-break entities
    \sloppy
    % Setup hyperref package
    \hypersetup{
      breaklinks=true,  % so long urls are correctly broken across lines
      colorlinks=true,
      urlcolor=urlcolor,
      linkcolor=linkcolor,
      citecolor=citecolor,
      }
    % Slightly bigger margins than the latex defaults
    
    \geometry{verbose,tmargin=1in,bmargin=1in,lmargin=1in,rmargin=1in}
    
    

\begin{document}
    
    \maketitle
    
    

    
    \hypertarget{banana-trade-relationships-among-countries}{%
\section{Banana Trade Relationships Among
Countries}\label{banana-trade-relationships-among-countries}}

\hypertarget{christopher-pedersen-and-dominic-frerichs}{%
\subsection{Christopher Pedersen and Dominic
Frerichs}\label{christopher-pedersen-and-dominic-frerichs}}

\hypertarget{professor-aguiar}{%
\subsection{Professor Aguiar}\label{professor-aguiar}}

\hypertarget{math1010-introduction-to-networks}{%
\subsection{MATH1010: Introduction to
Networks}\label{math1010-introduction-to-networks}}

    \hypertarget{importing-the-packages-well-use}{%
\subsection{Importing the packages we'll
use}\label{importing-the-packages-well-use}}

    \begin{tcolorbox}[breakable, size=fbox, boxrule=1pt, pad at break*=1mm,colback=cellbackground, colframe=cellborder]
\prompt{In}{incolor}{1}{\boxspacing}
\begin{Verbatim}[commandchars=\\\{\}]
\PY{c+c1}{\PYZsh{} Importing Packages}
\PY{k+kn}{import} \PY{n+nn}{numpy} \PY{k}{as} \PY{n+nn}{np}
\PY{k+kn}{import} \PY{n+nn}{networkx} \PY{k}{as} \PY{n+nn}{nx}
\PY{k+kn}{import} \PY{n+nn}{matplotlib}\PY{n+nn}{.}\PY{n+nn}{pyplot} \PY{k}{as} \PY{n+nn}{plt}
\PY{k+kn}{from} \PY{n+nn}{networkx}\PY{n+nn}{.}\PY{n+nn}{algorithms}\PY{n+nn}{.}\PY{n+nn}{community} \PY{k+kn}{import} \PY{n}{girvan\PYZus{}newman}
\PY{k+kn}{import} \PY{n+nn}{matplotlib}\PY{n+nn}{.}\PY{n+nn}{cm} \PY{k}{as} \PY{n+nn}{cmx}
\end{Verbatim}
\end{tcolorbox}

    \hypertarget{loading-in-the-banana-import-and-export-dataset-from-the-fao}{%
\subsection{Loading in the banana import and export dataset from the
FAO}\label{loading-in-the-banana-import-and-export-dataset-from-the-fao}}

    \begin{tcolorbox}[breakable, size=fbox, boxrule=1pt, pad at break*=1mm,colback=cellbackground, colframe=cellborder]
\prompt{In}{incolor}{2}{\boxspacing}
\begin{Verbatim}[commandchars=\\\{\}]
\PY{c+c1}{\PYZsh{} Loading in the dataset}
\PY{n}{folder} \PY{o}{=} \PY{l+s+s2}{\PYZdq{}}\PY{l+s+s2}{\PYZdq{}}
\PY{n}{file} \PY{o}{=} \PY{l+s+s2}{\PYZdq{}}\PY{l+s+s2}{FAO\PYZus{}Bananas}\PY{l+s+s2}{\PYZdq{}}
\PY{n}{ext} \PY{o}{=} \PY{l+s+s2}{\PYZdq{}}\PY{l+s+s2}{.npy}\PY{l+s+s2}{\PYZdq{}}
\PY{n}{FAO\PYZus{}Bananas} \PY{o}{=} \PY{n}{np}\PY{o}{.}\PY{n}{load}\PY{p}{(} \PY{n}{folder}\PY{o}{+}\PY{n}{file}\PY{o}{+}\PY{n}{ext} \PY{p}{)}

\PY{n+nb}{print}\PY{p}{(}\PY{n}{FAO\PYZus{}Bananas}\PY{p}{)}
\end{Verbatim}
\end{tcolorbox}

    \begin{Verbatim}[commandchars=\\\{\}]
[[0. 0. 0. {\ldots} 0. 0. 0.]
 [0. 0. 0. {\ldots} 1. 0. 0.]
 [0. 0. 0. {\ldots} 0. 1. 0.]
 {\ldots}
 [0. 0. 0. {\ldots} 0. 0. 0.]
 [0. 0. 0. {\ldots} 0. 0. 0.]
 [0. 0. 0. {\ldots} 0. 0. 0.]]
    \end{Verbatim}

    \hypertarget{creating-the-network-from-the-.npy-file}{%
\subsection{Creating the network from the .npy
file}\label{creating-the-network-from-the-.npy-file}}

    \begin{tcolorbox}[breakable, size=fbox, boxrule=1pt, pad at break*=1mm,colback=cellbackground, colframe=cellborder]
\prompt{In}{incolor}{3}{\boxspacing}
\begin{Verbatim}[commandchars=\\\{\}]
\PY{c+c1}{\PYZsh{} Creating the Network}
\PY{n}{FAO\PYZus{}Bananas\PYZus{}G} \PY{o}{=} \PY{n}{nx}\PY{o}{.}\PY{n}{to\PYZus{}networkx\PYZus{}graph}\PY{p}{(}\PY{n}{FAO\PYZus{}Bananas}\PY{p}{,} \PY{n}{create\PYZus{}using}\PY{o}{=}\PY{n}{nx}\PY{o}{.}\PY{n}{DiGraph}\PY{p}{)}
\end{Verbatim}
\end{tcolorbox}

    \hypertarget{context}{%
\subsection{Context}\label{context}}

    This network dataset originates from the Food and Agriculture
Organization of the United Nations (FAO). The FAO is an agency of the
United Nations that leads the effort to defeat hunger worldwide by
achieving food security for all. They collect data on the imports and
exports of different products between countries, including bananas which
we are focusing on today. The data represents banana trade relationships
between countries in 2010.

    \hypertarget{basic-network-attributes}{%
\subsection{Basic network attributes}\label{basic-network-attributes}}

First, we want to verify whether or not the network is directed. To do
this, we use the ``all\_close'' function. Because it returns ``False'',
it means the matrix and its transpose are not the same and therefore the
network is directed.

    \begin{tcolorbox}[breakable, size=fbox, boxrule=1pt, pad at break*=1mm,colback=cellbackground, colframe=cellborder]
\prompt{In}{incolor}{4}{\boxspacing}
\begin{Verbatim}[commandchars=\\\{\}]
\PY{n}{np}\PY{o}{.}\PY{n}{allclose}\PY{p}{(}\PY{n}{FAO\PYZus{}Bananas}\PY{p}{,} \PY{n}{FAO\PYZus{}Bananas}\PY{o}{.}\PY{n}{T}\PY{p}{)}
\end{Verbatim}
\end{tcolorbox}

            \begin{tcolorbox}[breakable, size=fbox, boxrule=.5pt, pad at break*=1mm, opacityfill=0]
\prompt{Out}{outcolor}{4}{\boxspacing}
\begin{Verbatim}[commandchars=\\\{\}]
False
\end{Verbatim}
\end{tcolorbox}
        
    \hypertarget{number-of-nodes-in-the-network}{%
\subsubsection{Number of nodes in the
network?}\label{number-of-nodes-in-the-network}}

    \begin{tcolorbox}[breakable, size=fbox, boxrule=1pt, pad at break*=1mm,colback=cellbackground, colframe=cellborder]
\prompt{In}{incolor}{5}{\boxspacing}
\begin{Verbatim}[commandchars=\\\{\}]
\PY{n}{np}\PY{o}{.}\PY{n}{shape}\PY{p}{(}\PY{n}{FAO\PYZus{}Bananas}\PY{p}{)}
\PY{n}{N} \PY{o}{=} \PY{n}{np}\PY{o}{.}\PY{n}{shape}\PY{p}{(}\PY{n}{FAO\PYZus{}Bananas}\PY{p}{)}\PY{p}{[}\PY{l+m+mi}{0}\PY{p}{]}
\PY{n+nb}{print}\PY{p}{(}\PY{n}{N}\PY{p}{)}
\end{Verbatim}
\end{tcolorbox}

    \begin{Verbatim}[commandchars=\\\{\}]
214
    \end{Verbatim}

    There are a total 214 nodes in the network, each representing a country
in the trade network.

    \hypertarget{number-of-edges-in-the-network}{%
\subsubsection{Number of edges in the
network?}\label{number-of-edges-in-the-network}}

    \begin{tcolorbox}[breakable, size=fbox, boxrule=1pt, pad at break*=1mm,colback=cellbackground, colframe=cellborder]
\prompt{In}{incolor}{6}{\boxspacing}
\begin{Verbatim}[commandchars=\\\{\}]
\PY{n+nb}{print}\PY{p}{(}\PY{n}{FAO\PYZus{}Bananas\PYZus{}G}\PY{o}{.}\PY{n}{number\PYZus{}of\PYZus{}edges}\PY{p}{(}\PY{p}{)}\PY{p}{)}
\end{Verbatim}
\end{tcolorbox}

    \begin{Verbatim}[commandchars=\\\{\}]
1752
    \end{Verbatim}

    There are a total 1,752 edges in the network, representing import and
export relationships of bananas among countries.

    \hypertarget{weighted-or-unweighted}{%
\subsubsection{Weighted or unweighted?}\label{weighted-or-unweighted}}

    Our network dataset is weighted. The edges are weighted and represent a
certain unit of bananas being imported and exported between countries.

    \hypertarget{connected-or-disconnected}{%
\subsubsection{Connected or
disconnected?}\label{connected-or-disconnected}}

    Because we found our banana network to be directed above, we will
determine strongly connected components.

    \begin{tcolorbox}[breakable, size=fbox, boxrule=1pt, pad at break*=1mm,colback=cellbackground, colframe=cellborder]
\prompt{In}{incolor}{7}{\boxspacing}
\begin{Verbatim}[commandchars=\\\{\}]
\PY{n+nb}{print}\PY{p}{(}\PY{n}{nx}\PY{o}{.}\PY{n}{number\PYZus{}strongly\PYZus{}connected\PYZus{}components}\PY{p}{(}\PY{n}{FAO\PYZus{}Bananas\PYZus{}G}\PY{p}{)}\PY{p}{)}
\PY{n+nb}{print}\PY{p}{(}\PY{n}{nx}\PY{o}{.}\PY{n}{is\PYZus{}strongly\PYZus{}connected}\PY{p}{(}\PY{n}{FAO\PYZus{}Bananas\PYZus{}G}\PY{p}{)}\PY{p}{)}
\end{Verbatim}
\end{tcolorbox}

    \begin{Verbatim}[commandchars=\\\{\}]
104
False
    \end{Verbatim}

    These results tell us that the overall network is not strongly connected
and has 104 separate strongly connected components.

    \hypertarget{plotting-the-network}{%
\subsection{Plotting the network}\label{plotting-the-network}}

    \hypertarget{customizing-our-visualization-with-country-labels}{%
\subsubsection{Customizing our visualization with country
labels}\label{customizing-our-visualization-with-country-labels}}

    \begin{tcolorbox}[breakable, size=fbox, boxrule=1pt, pad at break*=1mm,colback=cellbackground, colframe=cellborder]
\prompt{In}{incolor}{8}{\boxspacing}
\begin{Verbatim}[commandchars=\\\{\}]
\PY{c+c1}{\PYZsh{} Reading in country names}
\PY{n}{Names} \PY{o}{=} \PY{p}{[}\PY{p}{]}
\PY{k}{with} \PY{n+nb}{open}\PY{p}{(}\PY{l+s+s1}{\PYZsq{}}\PY{l+s+s1}{fao\PYZus{}trade\PYZus{}names.txt}\PY{l+s+s1}{\PYZsq{}}\PY{p}{,} \PY{l+s+s1}{\PYZsq{}}\PY{l+s+s1}{r}\PY{l+s+s1}{\PYZsq{}}\PY{p}{)} \PY{k}{as} \PY{n}{fd}\PY{p}{:}
    \PY{n}{lines} \PY{o}{=} \PY{n}{fd}\PY{o}{.}\PY{n}{readlines}\PY{p}{(}\PY{p}{)}
    \PY{k}{for} \PY{n}{line} \PY{o+ow}{in} \PY{n}{lines}\PY{p}{:}
        \PY{n}{name} \PY{o}{=} \PY{n}{line}\PY{o}{.}\PY{n}{split}\PY{p}{(}\PY{l+s+s1}{\PYZsq{}}\PY{l+s+se}{\PYZbs{}n}\PY{l+s+s1}{\PYZsq{}}\PY{p}{)}\PY{p}{[}\PY{l+m+mi}{0}\PY{p}{]}
        \PY{n}{Names}\PY{o}{.}\PY{n}{append}\PY{p}{(}\PY{n}{name}\PY{p}{)}
\end{Verbatim}
\end{tcolorbox}

    \begin{tcolorbox}[breakable, size=fbox, boxrule=1pt, pad at break*=1mm,colback=cellbackground, colframe=cellborder]
\prompt{In}{incolor}{9}{\boxspacing}
\begin{Verbatim}[commandchars=\\\{\}]
\PY{c+c1}{\PYZsh{} Creating dictionary}
\PY{n}{names\PYZus{}dictionary} \PY{o}{=} \PY{p}{\PYZob{}} \PY{n}{i} \PY{p}{:} \PY{n}{Names}\PY{p}{[}\PY{n}{i}\PY{p}{]} \PY{k}{for} \PY{n}{i} \PY{o+ow}{in} \PY{n+nb}{range}\PY{p}{(}\PY{l+m+mi}{0}\PY{p}{,} \PY{n+nb}{len}\PY{p}{(}\PY{n}{Names}\PY{p}{)} \PY{p}{)} \PY{p}{\PYZcb{}}
\end{Verbatim}
\end{tcolorbox}

    \begin{tcolorbox}[breakable, size=fbox, boxrule=1pt, pad at break*=1mm,colback=cellbackground, colframe=cellborder]
\prompt{In}{incolor}{10}{\boxspacing}
\begin{Verbatim}[commandchars=\\\{\}]
\PY{c+c1}{\PYZsh{} Relabeling nodes with country names}
\PY{n}{FAO\PYZus{}Bananas\PYZus{}G} \PY{o}{=} \PY{n}{nx}\PY{o}{.}\PY{n}{relabel\PYZus{}nodes}\PY{p}{(}\PY{n}{FAO\PYZus{}Bananas\PYZus{}G}\PY{p}{,} \PY{n}{names\PYZus{}dictionary}\PY{p}{)}
\end{Verbatim}
\end{tcolorbox}

    \begin{tcolorbox}[breakable, size=fbox, boxrule=1pt, pad at break*=1mm,colback=cellbackground, colframe=cellborder]
\prompt{In}{incolor}{11}{\boxspacing}
\begin{Verbatim}[commandchars=\\\{\}]
\PY{c+c1}{\PYZsh{} Positioning and drawing of network (names)}
\PY{n}{pos\PYZus{}names} \PY{o}{=} \PY{n}{nx}\PY{o}{.}\PY{n}{spring\PYZus{}layout}\PY{p}{(}\PY{n}{FAO\PYZus{}Bananas\PYZus{}G}\PY{p}{,} \PY{n}{k}\PY{o}{=}\PY{l+m+mi}{2}\PY{p}{)}

\PY{n}{nx}\PY{o}{.}\PY{n}{draw}\PY{p}{(}\PY{n}{FAO\PYZus{}Bananas\PYZus{}G}\PY{p}{,} \PY{n}{pos\PYZus{}names}\PY{p}{,} \PY{n}{node\PYZus{}color}\PY{o}{=}\PY{l+s+s1}{\PYZsq{}}\PY{l+s+s1}{c}\PY{l+s+s1}{\PYZsq{}}\PY{p}{,} \PY{n}{node\PYZus{}size}\PY{o}{=}\PY{l+m+mi}{100}\PY{p}{,} \PY{n}{edge\PYZus{}color}\PY{o}{=}\PY{l+s+s1}{\PYZsq{}}\PY{l+s+s1}{grey}\PY{l+s+s1}{\PYZsq{}}\PY{p}{,} \PY{n}{with\PYZus{}labels}\PY{o}{=}\PY{k+kc}{True}\PY{p}{,} \PY{n}{font\PYZus{}size}\PY{o}{=}\PY{l+m+mi}{6}\PY{p}{,} \PY{n}{width}\PY{o}{=}\PY{l+m+mf}{0.3}\PY{p}{)}
\end{Verbatim}
\end{tcolorbox}

    \begin{center}
    \adjustimage{max size={0.9\linewidth}{0.9\paperheight}}{output_28_0.png}
    \end{center}
    { \hspace*{\fill} \\}
    
    \hypertarget{graph-partitioning-using-the-girvan-newman-algorithm}{%
\subsection{Graph partitioning using the Girvan-Newman
algorithm}\label{graph-partitioning-using-the-girvan-newman-algorithm}}

    \hypertarget{including-a-helper-function-for-plotting-the-communities-in-our-network}{%
\subsubsection{Including a helper function for plotting the communities
in our
network}\label{including-a-helper-function-for-plotting-the-communities-in-our-network}}

    \begin{tcolorbox}[breakable, size=fbox, boxrule=1pt, pad at break*=1mm,colback=cellbackground, colframe=cellborder]
\prompt{In}{incolor}{12}{\boxspacing}
\begin{Verbatim}[commandchars=\\\{\}]
\PY{c+c1}{\PYZsh{} Function for plotting communities in the network}
\PY{k}{def} \PY{n+nf}{Plot\PYZus{}Comm}\PY{p}{(}\PY{n}{Network}\PY{p}{,} \PY{n}{C}\PY{p}{,} \PY{n}{position} \PY{o}{=} \PY{k+kc}{None}\PY{p}{)}\PY{p}{:}
    \PY{n}{cmap} \PY{o}{=} \PY{n}{cmx}\PY{o}{.}\PY{n}{get\PYZus{}cmap}\PY{p}{(}\PY{n}{name}\PY{o}{=}\PY{l+s+s1}{\PYZsq{}}\PY{l+s+s1}{rainbow}\PY{l+s+s1}{\PYZsq{}}\PY{p}{)}
    \PY{n}{N} \PY{o}{=} \PY{n+nb}{len}\PY{p}{(}\PY{n}{Network}\PY{o}{.}\PY{n}{nodes}\PY{p}{(}\PY{p}{)}\PY{p}{)}
    \PY{n}{K} \PY{o}{=} \PY{n+nb}{len}\PY{p}{(}\PY{n}{C}\PY{p}{)}
    \PY{n}{color\PYZus{}map} \PY{o}{=} \PY{p}{[}\PY{l+s+s1}{\PYZsq{}}\PY{l+s+s1}{k}\PY{l+s+s1}{\PYZsq{}}\PY{p}{]}\PY{o}{*}\PY{n}{N}
    \PY{k}{for} \PY{n}{i} \PY{o+ow}{in} \PY{n+nb}{range}\PY{p}{(}\PY{n}{K}\PY{p}{)}\PY{p}{:}
        \PY{k}{for} \PY{n}{j} \PY{o+ow}{in} \PY{n+nb}{range}\PY{p}{(}\PY{n+nb}{len}\PY{p}{(}\PY{n}{C}\PY{p}{[}\PY{n}{i}\PY{p}{]}\PY{p}{)}\PY{p}{)}\PY{p}{:}
            \PY{n}{color\PYZus{}map}\PY{p}{[} \PY{n}{C}\PY{p}{[}\PY{n}{i}\PY{p}{]}\PY{p}{[}\PY{n}{j}\PY{p}{]} \PY{p}{]} \PY{o}{=} \PY{n}{cmap}\PY{p}{(}\PY{n}{i}\PY{o}{/}\PY{n}{K}\PY{p}{)}
    \PY{k}{if} \PY{n}{position} \PY{o+ow}{is} \PY{k+kc}{None}\PY{p}{:}
        \PY{n}{pos} \PY{o}{=} \PY{n}{nx}\PY{o}{.}\PY{n}{spring\PYZus{}layout}\PY{p}{(}\PY{n}{Network}\PY{p}{,} \PY{n}{k}\PY{o}{=}\PY{l+m+mf}{0.25}\PY{p}{,}\PY{n}{iterations}\PY{o}{=}\PY{l+m+mi}{20}\PY{p}{)}
    \PY{k}{else}\PY{p}{:}
        \PY{n}{pos} \PY{o}{=} \PY{n}{position}
    \PY{n}{fig} \PY{o}{=} \PY{n}{plt}\PY{o}{.}\PY{n}{figure}\PY{p}{(}\PY{p}{)}
    \PY{n}{nx}\PY{o}{.}\PY{n}{draw}\PY{p}{(}\PY{n}{Network}\PY{p}{,} \PY{n}{pos}\PY{p}{,} \PY{n}{node\PYZus{}color}\PY{o}{=}\PY{n}{color\PYZus{}map}\PY{p}{,} \PY{n}{node\PYZus{}size}\PY{o}{=}\PY{l+m+mi}{100}\PY{p}{,} \PY{n}{edge\PYZus{}color}\PY{o}{=}\PY{l+s+s1}{\PYZsq{}}\PY{l+s+s1}{grey}\PY{l+s+s1}{\PYZsq{}}\PY{p}{,} \PY{n}{with\PYZus{}labels}\PY{o}{=}\PY{k+kc}{True}\PY{p}{,} \PY{n}{font\PYZus{}size}\PY{o}{=}\PY{l+m+mi}{6}\PY{p}{,} \PY{n}{width}\PY{o}{=}\PY{l+m+mf}{0.3}\PY{p}{)}
    \PY{n}{plt}\PY{o}{.}\PY{n}{show}\PY{p}{(}\PY{p}{)}
    \PY{k}{return}
\end{Verbatim}
\end{tcolorbox}

    \begin{tcolorbox}[breakable, size=fbox, boxrule=1pt, pad at break*=1mm,colback=cellbackground, colframe=cellborder]
\prompt{In}{incolor}{13}{\boxspacing}
\begin{Verbatim}[commandchars=\\\{\}]
\PY{c+c1}{\PYZsh{} Helper code for plotting communities with country names}
\PY{n}{comm} \PY{o}{=} \PY{n}{girvan\PYZus{}newman}\PY{p}{(}\PY{n}{FAO\PYZus{}Bananas\PYZus{}G}\PY{p}{)}
\PY{n}{gn\PYZus{}communities} \PY{o}{=} \PY{n+nb}{tuple}\PY{p}{(}\PY{n+nb}{sorted}\PY{p}{(}\PY{n}{c}\PY{p}{)} \PY{k}{for} \PY{n}{c} \PY{o+ow}{in} \PY{n+nb}{next}\PY{p}{(}\PY{n}{comm}\PY{p}{)}\PY{p}{)}

\PY{k}{def} \PY{n+nf}{switch\PYZus{}to\PYZus{}numbers}\PY{p}{(}\PY{n}{G}\PY{p}{,} \PY{n}{C}\PY{p}{)}\PY{p}{:}
    \PY{n}{num\PYZus{}comm} \PY{o}{=} \PY{n+nb}{len}\PY{p}{(}\PY{n}{C}\PY{p}{)}
    \PY{n}{a} \PY{o}{=} \PY{p}{[}\PY{p}{]}
    \PY{k}{for} \PY{n}{c} \PY{o+ow}{in} \PY{n+nb}{range}\PY{p}{(}\PY{n}{num\PYZus{}comm}\PY{p}{)}\PY{p}{:}
        \PY{n}{a}\PY{o}{.}\PY{n}{append}\PY{p}{(}\PY{p}{[}\PY{l+m+mi}{0}\PY{p}{]}\PY{p}{)}
    \PY{k}{for} \PY{n}{i}\PY{p}{,} \PY{n}{node} \PY{o+ow}{in} \PY{n+nb}{enumerate}\PY{p}{(}\PY{n}{G}\PY{o}{.}\PY{n}{nodes}\PY{p}{(}\PY{p}{)}\PY{p}{)}\PY{p}{:}
        \PY{k}{for} \PY{n}{j} \PY{o+ow}{in} \PY{n+nb}{range}\PY{p}{(}\PY{n}{num\PYZus{}comm}\PY{p}{)}\PY{p}{:}
            \PY{k}{if} \PY{n}{node} \PY{o+ow}{in} \PY{n}{C}\PY{p}{[}\PY{n}{j}\PY{p}{]}\PY{p}{:}
                \PY{n}{a}\PY{p}{[}\PY{n}{j}\PY{p}{]}\PY{o}{.}\PY{n}{append}\PY{p}{(}\PY{n}{i}\PY{p}{)}
    \PY{k}{for} \PY{n}{c} \PY{o+ow}{in} \PY{n+nb}{range}\PY{p}{(}\PY{n}{num\PYZus{}comm}\PY{p}{)}\PY{p}{:}
        \PY{n}{a}\PY{p}{[}\PY{n}{c}\PY{p}{]} \PY{o}{=} \PY{n}{a}\PY{p}{[}\PY{n}{c}\PY{p}{]}\PY{p}{[}\PY{l+m+mi}{1}\PY{p}{:}\PY{p}{]}
    \PY{k}{return} \PY{n}{a}
\end{Verbatim}
\end{tcolorbox}

    \hypertarget{plotting-the-communities-in-the-network}{%
\subsubsection{Plotting the communities in the
network}\label{plotting-the-communities-in-the-network}}

    \begin{tcolorbox}[breakable, size=fbox, boxrule=1pt, pad at break*=1mm,colback=cellbackground, colframe=cellborder]
\prompt{In}{incolor}{14}{\boxspacing}
\begin{Verbatim}[commandchars=\\\{\}]
\PY{c+c1}{\PYZsh{} Plotting network with communites and country names}
\PY{n}{named\PYZus{}communities} \PY{o}{=} \PY{n}{switch\PYZus{}to\PYZus{}numbers}\PY{p}{(}\PY{n}{FAO\PYZus{}Bananas\PYZus{}G}\PY{p}{,} \PY{n}{gn\PYZus{}communities}\PY{p}{)}

\PY{n}{Plot\PYZus{}Comm}\PY{p}{(}\PY{n}{FAO\PYZus{}Bananas\PYZus{}G}\PY{p}{,} \PY{n}{named\PYZus{}communities}\PY{p}{,} \PY{n}{pos\PYZus{}names}\PY{p}{)}
\end{Verbatim}
\end{tcolorbox}

    \begin{center}
    \adjustimage{max size={0.9\linewidth}{0.9\paperheight}}{output_34_0.png}
    \end{center}
    { \hspace*{\fill} \\}
    
    The communities identified by the Girvan-Newman algorithm do make sense
given the context of our data. When we look at some of the countries
excluded from the largest component of the network, it makes sense that
they are not importing or exporting bananas (i.e., North Korea).

    \hypertarget{centrality-in-the-banana-network}{%
\subsection{Centrality in the banana
network}\label{centrality-in-the-banana-network}}

    \begin{tcolorbox}[breakable, size=fbox, boxrule=1pt, pad at break*=1mm,colback=cellbackground, colframe=cellborder]
\prompt{In}{incolor}{15}{\boxspacing}
\begin{Verbatim}[commandchars=\\\{\}]
\PY{n}{deg\PYZus{}cen} \PY{o}{=} \PY{n}{nx}\PY{o}{.}\PY{n}{degree\PYZus{}centrality}\PY{p}{(}\PY{n}{FAO\PYZus{}Bananas\PYZus{}G}\PY{p}{)}
\PY{n}{close\PYZus{}cen} \PY{o}{=} \PY{n}{nx}\PY{o}{.}\PY{n}{closeness\PYZus{}centrality}\PY{p}{(}\PY{n}{FAO\PYZus{}Bananas\PYZus{}G}\PY{p}{)}
\PY{n}{eigen\PYZus{}cen} \PY{o}{=} \PY{n}{nx}\PY{o}{.}\PY{n}{eigenvector\PYZus{}centrality}\PY{p}{(}\PY{n}{FAO\PYZus{}Bananas\PYZus{}G}\PY{p}{)}
\PY{n}{betw\PYZus{}cen} \PY{o}{=} \PY{n}{nx}\PY{o}{.}\PY{n}{betweenness\PYZus{}centrality}\PY{p}{(}\PY{n}{FAO\PYZus{}Bananas\PYZus{}G}\PY{p}{)}

\PY{n+nb}{print}\PY{p}{(}\PY{l+s+s1}{\PYZsq{}}\PY{l+s+s1}{The country with highest degree centrality is}\PY{l+s+s1}{\PYZsq{}}\PY{p}{,} \PY{n+nb}{max}\PY{p}{(}\PY{n}{deg\PYZus{}cen}\PY{p}{,} \PY{n}{key}\PY{o}{=}\PY{k}{lambda} \PY{n}{key}\PY{p}{:} \PY{n}{deg\PYZus{}cen}\PY{p}{[}\PY{n}{key}\PY{p}{]}\PY{p}{)}\PY{p}{)}
\PY{n+nb}{print}\PY{p}{(}\PY{l+s+s1}{\PYZsq{}}\PY{l+s+s1}{The country with highest closeness centrality is}\PY{l+s+s1}{\PYZsq{}}\PY{p}{,} \PY{n+nb}{max}\PY{p}{(}\PY{n}{close\PYZus{}cen}\PY{p}{,} \PY{n}{key}\PY{o}{=}\PY{k}{lambda} \PY{n}{key}\PY{p}{:} \PY{n}{close\PYZus{}cen}\PY{p}{[}\PY{n}{key}\PY{p}{]}\PY{p}{)}\PY{p}{)}
\PY{n+nb}{print}\PY{p}{(}\PY{l+s+s1}{\PYZsq{}}\PY{l+s+s1}{The country with highest eigenvector centrality is}\PY{l+s+s1}{\PYZsq{}}\PY{p}{,} \PY{n+nb}{max}\PY{p}{(}\PY{n}{eigen\PYZus{}cen}\PY{p}{,} \PY{n}{key}\PY{o}{=}\PY{k}{lambda} \PY{n}{key}\PY{p}{:} \PY{n}{eigen\PYZus{}cen}\PY{p}{[}\PY{n}{key}\PY{p}{]}\PY{p}{)}\PY{p}{)}
\PY{n+nb}{print}\PY{p}{(}\PY{l+s+s1}{\PYZsq{}}\PY{l+s+s1}{The country with highest betweenness centrality is}\PY{l+s+s1}{\PYZsq{}}\PY{p}{,} \PY{n+nb}{max}\PY{p}{(}\PY{n}{betw\PYZus{}cen}\PY{p}{,} \PY{n}{key}\PY{o}{=}\PY{k}{lambda} \PY{n}{key}\PY{p}{:} \PY{n}{betw\PYZus{}cen}\PY{p}{[}\PY{n}{key}\PY{p}{]}\PY{p}{)}\PY{p}{)}
\end{Verbatim}
\end{tcolorbox}

    \begin{Verbatim}[commandchars=\\\{\}]
The country with highest degree centrality is Germany
The country with highest closeness centrality is China
The country with highest eigenvector centrality is Germany
The country with highest betweenness centrality is Germany
    \end{Verbatim}

    \hypertarget{centrality-graph-closeness}{%
\subsubsection{Centrality Graph
(closeness)}\label{centrality-graph-closeness}}

    \begin{tcolorbox}[breakable, size=fbox, boxrule=1pt, pad at break*=1mm,colback=cellbackground, colframe=cellborder]
\prompt{In}{incolor}{16}{\boxspacing}
\begin{Verbatim}[commandchars=\\\{\}]
\PY{n}{nx}\PY{o}{.}\PY{n}{draw}\PY{p}{(}\PY{n}{FAO\PYZus{}Bananas\PYZus{}G}\PY{p}{,} \PY{n}{pos\PYZus{}names}\PY{p}{,} \PY{n}{node\PYZus{}color}\PY{o}{=}\PY{l+s+s1}{\PYZsq{}}\PY{l+s+s1}{c}\PY{l+s+s1}{\PYZsq{}}\PY{p}{,} \PY{n}{node\PYZus{}size}\PY{o}{=}\PY{l+m+mi}{100}\PY{p}{,} \PY{n}{edge\PYZus{}color}\PY{o}{=}\PY{l+s+s1}{\PYZsq{}}\PY{l+s+s1}{grey}\PY{l+s+s1}{\PYZsq{}}\PY{p}{,} \PY{n}{with\PYZus{}labels}\PY{o}{=}\PY{k+kc}{False}\PY{p}{,} \PY{n}{width}\PY{o}{=}\PY{l+m+mf}{0.3}\PY{p}{)}
\PY{n}{nx}\PY{o}{.}\PY{n}{draw\PYZus{}networkx\PYZus{}nodes}\PY{p}{(}\PY{n}{FAO\PYZus{}Bananas\PYZus{}G}\PY{p}{,} \PY{n}{pos\PYZus{}names}\PY{p}{,} \PY{n}{node\PYZus{}color} \PY{o}{=} \PY{l+s+s1}{\PYZsq{}}\PY{l+s+s1}{red}\PY{l+s+s1}{\PYZsq{}}\PY{p}{,} \PY{n}{nodelist} \PY{o}{=} \PY{p}{[}\PY{l+s+s1}{\PYZsq{}}\PY{l+s+s1}{China}\PY{l+s+s1}{\PYZsq{}}\PY{p}{]}\PY{p}{,} \PY{n}{label} \PY{o}{=} \PY{l+s+s1}{\PYZsq{}}\PY{l+s+s1}{China}\PY{l+s+s1}{\PYZsq{}}\PY{p}{,} \PY{n}{node\PYZus{}size}\PY{o}{=}\PY{l+m+mi}{100}\PY{p}{,} \PY{n}{node\PYZus{}shape}\PY{o}{=}\PY{l+s+s1}{\PYZsq{}}\PY{l+s+s1}{s}\PY{l+s+s1}{\PYZsq{}}\PY{p}{)}
\end{Verbatim}
\end{tcolorbox}

            \begin{tcolorbox}[breakable, size=fbox, boxrule=.5pt, pad at break*=1mm, opacityfill=0]
\prompt{Out}{outcolor}{16}{\boxspacing}
\begin{Verbatim}[commandchars=\\\{\}]
<matplotlib.collections.PathCollection at 0x1d56c70ef80>
\end{Verbatim}
\end{tcolorbox}
        
    \begin{center}
    \adjustimage{max size={0.9\linewidth}{0.9\paperheight}}{output_39_1.png}
    \end{center}
    { \hspace*{\fill} \\}
    
    \hypertarget{centrality-graph-degree-eigenvenctor-betweenness}{%
\subsubsection{Centrality Graph (degree, eigenvenctor,
betweenness)}\label{centrality-graph-degree-eigenvenctor-betweenness}}

    \begin{tcolorbox}[breakable, size=fbox, boxrule=1pt, pad at break*=1mm,colback=cellbackground, colframe=cellborder]
\prompt{In}{incolor}{17}{\boxspacing}
\begin{Verbatim}[commandchars=\\\{\}]
\PY{n}{nx}\PY{o}{.}\PY{n}{draw}\PY{p}{(}\PY{n}{FAO\PYZus{}Bananas\PYZus{}G}\PY{p}{,} \PY{n}{pos\PYZus{}names}\PY{p}{,} \PY{n}{node\PYZus{}color}\PY{o}{=}\PY{l+s+s1}{\PYZsq{}}\PY{l+s+s1}{c}\PY{l+s+s1}{\PYZsq{}}\PY{p}{,} \PY{n}{node\PYZus{}size}\PY{o}{=}\PY{l+m+mi}{100}\PY{p}{,} \PY{n}{edge\PYZus{}color}\PY{o}{=}\PY{l+s+s1}{\PYZsq{}}\PY{l+s+s1}{grey}\PY{l+s+s1}{\PYZsq{}}\PY{p}{,} \PY{n}{with\PYZus{}labels}\PY{o}{=}\PY{k+kc}{False}\PY{p}{,} \PY{n}{width}\PY{o}{=}\PY{l+m+mf}{0.3}\PY{p}{)}
\PY{n}{nx}\PY{o}{.}\PY{n}{draw\PYZus{}networkx\PYZus{}nodes}\PY{p}{(}\PY{n}{FAO\PYZus{}Bananas\PYZus{}G}\PY{p}{,} \PY{n}{pos\PYZus{}names}\PY{p}{,} \PY{n}{node\PYZus{}color} \PY{o}{=} \PY{l+s+s1}{\PYZsq{}}\PY{l+s+s1}{red}\PY{l+s+s1}{\PYZsq{}}\PY{p}{,} \PY{n}{nodelist} \PY{o}{=} \PY{p}{[}\PY{l+s+s1}{\PYZsq{}}\PY{l+s+s1}{Germany}\PY{l+s+s1}{\PYZsq{}}\PY{p}{]}\PY{p}{,} \PY{n}{label} \PY{o}{=} \PY{l+s+s1}{\PYZsq{}}\PY{l+s+s1}{Germany}\PY{l+s+s1}{\PYZsq{}}\PY{p}{,} \PY{n}{node\PYZus{}size}\PY{o}{=}\PY{l+m+mi}{100}\PY{p}{,} \PY{n}{node\PYZus{}shape}\PY{o}{=}\PY{l+s+s1}{\PYZsq{}}\PY{l+s+s1}{s}\PY{l+s+s1}{\PYZsq{}}\PY{p}{)}
\end{Verbatim}
\end{tcolorbox}

            \begin{tcolorbox}[breakable, size=fbox, boxrule=.5pt, pad at break*=1mm, opacityfill=0]
\prompt{Out}{outcolor}{17}{\boxspacing}
\begin{Verbatim}[commandchars=\\\{\}]
<matplotlib.collections.PathCollection at 0x1d56d065f00>
\end{Verbatim}
\end{tcolorbox}
        
    \begin{center}
    \adjustimage{max size={0.9\linewidth}{0.9\paperheight}}{output_41_1.png}
    \end{center}
    { \hspace*{\fill} \\}
    
    \hypertarget{centrality-explanations}{%
\subsubsection{Centrality explanations}\label{centrality-explanations}}

    Yes, the centrality metrics did identify different nodes as the ``most
central'' node. Germany was identified as having the highest degree,
eigenvector, and betweenness centrality while China was identified as
having the highest closeness centrality.

Given the nature of the banana trade network, we think that degree
centrality is most informative. Degree centrality defines the important
nodes to be those that are connected to lots of other nodes. In the
context of our network, this means that the country with the highest
degree centrality is importing and exporting bananas to the most other
countries. This establishes their importance in the network overall.

    \hypertarget{clustering-coefficient}{%
\subsection{Clustering coefficient}\label{clustering-coefficient}}

    \begin{tcolorbox}[breakable, size=fbox, boxrule=1pt, pad at break*=1mm,colback=cellbackground, colframe=cellborder]
\prompt{In}{incolor}{18}{\boxspacing}
\begin{Verbatim}[commandchars=\\\{\}]
\PY{n}{clustering} \PY{o}{=} \PY{n}{nx}\PY{o}{.}\PY{n}{algorithms}\PY{o}{.}\PY{n}{clustering}\PY{p}{(}\PY{n}{FAO\PYZus{}Bananas\PYZus{}G}\PY{p}{)}
\PY{n+nb}{print}\PY{p}{(}\PY{l+s+s1}{\PYZsq{}}\PY{l+s+s1}{The country with lowest clustering coefficient is}\PY{l+s+s1}{\PYZsq{}}\PY{p}{,} \PY{n+nb}{min}\PY{p}{(}\PY{n}{clustering}\PY{p}{,} \PY{n}{key}\PY{o}{=}\PY{k}{lambda} \PY{n}{key}\PY{p}{:} \PY{n}{clustering}\PY{p}{[}\PY{n}{key}\PY{p}{]}\PY{p}{)}\PY{p}{)}

\PY{n}{average\PYZus{}cc} \PY{o}{=} \PY{n}{nx}\PY{o}{.}\PY{n}{algorithms}\PY{o}{.}\PY{n}{average\PYZus{}clustering}\PY{p}{(}\PY{n}{FAO\PYZus{}Bananas\PYZus{}G}\PY{p}{)}
\PY{n+nb}{print}\PY{p}{(}\PY{l+s+s1}{\PYZsq{}}\PY{l+s+s1}{The average clustering coefficient for the banana network is }\PY{l+s+s1}{\PYZsq{}}\PY{p}{,} \PY{n}{average\PYZus{}cc}\PY{p}{)}
\end{Verbatim}
\end{tcolorbox}

    \begin{Verbatim}[commandchars=\\\{\}]
The country with lowest clustering coefficient is Afghanistan
The average clustering coefficient for the banana network is
0.19276834633257203
    \end{Verbatim}

    \begin{tcolorbox}[breakable, size=fbox, boxrule=1pt, pad at break*=1mm,colback=cellbackground, colframe=cellborder]
\prompt{In}{incolor}{19}{\boxspacing}
\begin{Verbatim}[commandchars=\\\{\}]
\PY{n}{nx}\PY{o}{.}\PY{n}{draw}\PY{p}{(}\PY{n}{FAO\PYZus{}Bananas\PYZus{}G}\PY{p}{,} \PY{n}{pos\PYZus{}names}\PY{p}{,} \PY{n}{node\PYZus{}color}\PY{o}{=}\PY{l+s+s1}{\PYZsq{}}\PY{l+s+s1}{c}\PY{l+s+s1}{\PYZsq{}}\PY{p}{,} \PY{n}{node\PYZus{}size}\PY{o}{=}\PY{l+m+mi}{100}\PY{p}{,} \PY{n}{edge\PYZus{}color}\PY{o}{=}\PY{l+s+s1}{\PYZsq{}}\PY{l+s+s1}{grey}\PY{l+s+s1}{\PYZsq{}}\PY{p}{,} \PY{n}{with\PYZus{}labels}\PY{o}{=}\PY{k+kc}{False}\PY{p}{,} \PY{n}{width}\PY{o}{=}\PY{l+m+mf}{0.3}\PY{p}{)}
\PY{n}{nx}\PY{o}{.}\PY{n}{draw\PYZus{}networkx\PYZus{}nodes}\PY{p}{(}\PY{n}{FAO\PYZus{}Bananas\PYZus{}G}\PY{p}{,} \PY{n}{pos\PYZus{}names}\PY{p}{,} \PY{n}{node\PYZus{}color} \PY{o}{=} \PY{l+s+s1}{\PYZsq{}}\PY{l+s+s1}{red}\PY{l+s+s1}{\PYZsq{}}\PY{p}{,} \PY{n}{nodelist} \PY{o}{=} \PY{p}{[}\PY{l+s+s1}{\PYZsq{}}\PY{l+s+s1}{Afghanistan}\PY{l+s+s1}{\PYZsq{}}\PY{p}{]}\PY{p}{,} \PY{n}{label} \PY{o}{=} \PY{l+s+s1}{\PYZsq{}}\PY{l+s+s1}{Afghanistan}\PY{l+s+s1}{\PYZsq{}}\PY{p}{,} \PY{n}{node\PYZus{}size}\PY{o}{=}\PY{l+m+mi}{100}\PY{p}{,} \PY{n}{node\PYZus{}shape}\PY{o}{=}\PY{l+s+s1}{\PYZsq{}}\PY{l+s+s1}{s}\PY{l+s+s1}{\PYZsq{}}\PY{p}{)}
\end{Verbatim}
\end{tcolorbox}

            \begin{tcolorbox}[breakable, size=fbox, boxrule=.5pt, pad at break*=1mm, opacityfill=0]
\prompt{Out}{outcolor}{19}{\boxspacing}
\begin{Verbatim}[commandchars=\\\{\}]
<matplotlib.collections.PathCollection at 0x1d56e8d5ae0>
\end{Verbatim}
\end{tcolorbox}
        
    \begin{center}
    \adjustimage{max size={0.9\linewidth}{0.9\paperheight}}{output_46_1.png}
    \end{center}
    { \hspace*{\fill} \\}
    
    These results show that Afghanistan has the lowest probability that any
pair of the countries it trades bananas with trade bananas with one
another (of the countries connected to other banana trading countries).
Furthermore, the average probability for a country in the network having
any pair of trading partners who trade with one another is 19.28\%.

    \hypertarget{degree-distribution}{%
\subsection{Degree Distribution}\label{degree-distribution}}

    \hypertarget{in-degree}{%
\subsubsection{In-Degree}\label{in-degree}}

    \begin{tcolorbox}[breakable, size=fbox, boxrule=1pt, pad at break*=1mm,colback=cellbackground, colframe=cellborder]
\prompt{In}{incolor}{20}{\boxspacing}
\begin{Verbatim}[commandchars=\\\{\}]
\PY{n}{degree\PYZus{}sequence\PYZus{}in} \PY{o}{=} \PY{n+nb}{sorted}\PY{p}{(}\PY{p}{(}\PY{n}{d} \PY{k}{for} \PY{n}{n}\PY{p}{,} \PY{n}{d} \PY{o+ow}{in} \PY{n}{FAO\PYZus{}Bananas\PYZus{}G}\PY{o}{.}\PY{n}{in\PYZus{}degree}\PY{p}{(}\PY{p}{)}\PY{p}{)}\PY{p}{,} \PY{n}{reverse}\PY{o}{=}\PY{k+kc}{True}\PY{p}{)}
\PY{n}{plt}\PY{o}{.}\PY{n}{bar}\PY{p}{(}\PY{o}{*}\PY{n}{np}\PY{o}{.}\PY{n}{unique}\PY{p}{(}\PY{n}{degree\PYZus{}sequence\PYZus{}in}\PY{p}{,} \PY{n}{return\PYZus{}counts}\PY{o}{=}\PY{k+kc}{True}\PY{p}{)}\PY{p}{)}
\end{Verbatim}
\end{tcolorbox}

            \begin{tcolorbox}[breakable, size=fbox, boxrule=.5pt, pad at break*=1mm, opacityfill=0]
\prompt{Out}{outcolor}{20}{\boxspacing}
\begin{Verbatim}[commandchars=\\\{\}]
<BarContainer object of 37 artists>
\end{Verbatim}
\end{tcolorbox}
        
    \begin{center}
    \adjustimage{max size={0.9\linewidth}{0.9\paperheight}}{output_50_1.png}
    \end{center}
    { \hspace*{\fill} \\}
    
    The in-degree of a node is the number of edges pointing to the node from
another. In this case, it is the number of countries a country imports
bananas from. We can see a right-skew to the graph, suggesting many
countries do not import bananas from any other country in the network.

    \hypertarget{out-degree}{%
\subsubsection{Out-Degree}\label{out-degree}}

    \begin{tcolorbox}[breakable, size=fbox, boxrule=1pt, pad at break*=1mm,colback=cellbackground, colframe=cellborder]
\prompt{In}{incolor}{21}{\boxspacing}
\begin{Verbatim}[commandchars=\\\{\}]
\PY{n}{degree\PYZus{}sequence\PYZus{}out} \PY{o}{=} \PY{n+nb}{sorted}\PY{p}{(}\PY{p}{(}\PY{n}{d} \PY{k}{for} \PY{n}{n}\PY{p}{,} \PY{n}{d} \PY{o+ow}{in} \PY{n}{FAO\PYZus{}Bananas\PYZus{}G}\PY{o}{.}\PY{n}{out\PYZus{}degree}\PY{p}{(}\PY{p}{)}\PY{p}{)}\PY{p}{,} \PY{n}{reverse}\PY{o}{=}\PY{k+kc}{True}\PY{p}{)}
\PY{n}{plt}\PY{o}{.}\PY{n}{bar}\PY{p}{(}\PY{o}{*}\PY{n}{np}\PY{o}{.}\PY{n}{unique}\PY{p}{(}\PY{n}{degree\PYZus{}sequence\PYZus{}out}\PY{p}{,} \PY{n}{return\PYZus{}counts}\PY{o}{=}\PY{k+kc}{True}\PY{p}{)}\PY{p}{)}
\end{Verbatim}
\end{tcolorbox}

            \begin{tcolorbox}[breakable, size=fbox, boxrule=.5pt, pad at break*=1mm, opacityfill=0]
\prompt{Out}{outcolor}{21}{\boxspacing}
\begin{Verbatim}[commandchars=\\\{\}]
<BarContainer object of 43 artists>
\end{Verbatim}
\end{tcolorbox}
        
    \begin{center}
    \adjustimage{max size={0.9\linewidth}{0.9\paperheight}}{output_53_1.png}
    \end{center}
    { \hspace*{\fill} \\}
    
    The out-degree of a node is the number of edges pointing from a node to
another. In this case, it is the number of countries a country exports
bananas to. We can see a right-skew to the graph, suggesting many
countries do not export bananas to any other country in the network.

    \hypertarget{conclusion}{%
\subsection{Conclusion}\label{conclusion}}

    Before performing an analysis of this dataset, we thought bananas were
exported from a select few countries that produced much of the world's
supply. Looking at the out-degree graph, we can see that while there are
many countries that do not export bananas, there is a sizeable number of
countries that still export a notable amount of bananas themselves.
There do not appear to be many countries that completely dominate the
banana export business.

Within the network, there are a lot of countries that are not
participating in the import or export of bananas. Therefore, we can
conclude that these countries either choose/are forced not to
import/export, can't afford to import/export, or produce their own
bananas. The number of non-participating countries is also much larger
than we expected.

In regards to desirable knowledge to better graph the network, we would
be interested in including information on trade restrictions between
countries that may prevent some countries from participating in the
import/export of food products like bananas. This could help explain why
there are so many non-participating countries like we mentioned above.
In terms of metadata, we would like information on quantities of bananas
being transported between countries, whether that be tons, number of
bananas, or shipping containers. Furthermore, GDP data per country could
be helpful to determine the significance of the banana trade market to a
country's economy.

    \hypertarget{references}{%
\subsection{References}\label{references}}

    ``About FAO.'' n.d. Food and Agriculture Organization of the United
Nations. Accessed February 1, 2023. http://www.fao.org/about/en/.

De Domenico, Manlio, Vincenzo Nicosia, Alexandre Arenas, and Vito
Latora. 2015. ``Structural Reducibility of Multilayer Networks.'' Nature
Communications 6 (1): 6864. https://doi.org/10.1038/ncomms7864.


    % Add a bibliography block to the postdoc
    
    
    
\end{document}
